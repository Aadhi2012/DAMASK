% ########################
\chapter{Crystallographic orientations}
% ########################

\section{Bunge Euler angles}
\label{bunges}
Euler angles $(\varphi_1, \phi, \varphi_2)$---following the Bunge convention---rotate the sample coordinate system ($X$, $Y$, $Z$  or RD, TD, ND) into the crystal coordinate system ($x_\text c$, $y_\text c$, $z_\text c$). 
Three successive rotations are carried out in the following way \citep[p.~4]{Bunge1982}:
\begin{enumerate}
	\item Rotate by $\varphi_1$ around Z, to bring X into the $x_\text c$--$y_\text c$-plane. The new intermediate axes are $X^\prime$,  $Y^\prime$ and $Z$ (unchanged).
	\item Now rotate by $\phi$ around  $X^\prime$, to make $Z$ parallel with $z_\text c$. The intermediate axes are  $X^\prime$,  $Y^{\prime\prime}$,  $Z^\prime$.
	\item A final rotation by $\varphi_2$ around  $Z^\prime \equiv z_\text c$ makes the rotated axes then identical to the crystal axes.
\end{enumerate}

The rotation matrix can be calculated as
\[% Gottstein pg 55
\tnsr{g}=\left(\begin{array}{ccc}
\cos{\varphi_1}\cos{\varphi_2}-\sin{\varphi_1}\sin{\varphi_2}\cos{\phi} & \sin{\varphi_1}\cos{\varphi_2}+\cos{\varphi_1}\sin{\varphi_2}\cos{\phi}& \sin{\varphi_2}\sin{\phi}\\
-\cos{\varphi_1}\sin{\varphi_2}-\sin{\varphi_1}\cos{\varphi_2}\cos{\phi} & -\sin{\varphi_1}\cos{\varphi_2}+\cos{\varphi_1}\cos{\varphi_2}\cos{\phi}& \cos{\varphi_2}\sin{\phi}\\
\sin{\varphi_1}\sin{\phi} & -\cos{\varphi_1}\sin{\phi}& \cos{\phi}
\end{array}\right)
\]

