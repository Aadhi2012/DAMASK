\section{Accessing the version controlled subroutine}
% This section is copied from the msuwiki: http://msusrv4/msuwiki/Theory%20and%20Simulation/SVN
Before you start: Before you are able to access the version-controlled software, you need to get a valid login to the msuhp9 server. Please ask either Berthold Becksch\"afer (-922) or Achim Kuhl (-923) to set up your permissions accordingly. 

\subsection{Windows}
\subsubsection{Putty}
Get yourself PuTTY and PuTTYgen from http://www.chiark.greenend.org.uk/~sgtatham/putty/download.html 
Generate a RSA (SSH2) 2048 bit strong key pair with PuTTYgen. Save the private part of the key to a secure location (My Documents or such). Copy the public part from the PuTTYgen window, paste it into a text-editor and save. Append the contents of that file to \verb|~/.ssh/authorized_keys| on any workstation you can log on to. 
Create a profile in PuTTY called "msuhp9" with host: msuhp9.mpie.de, your standard "MPIE\\myName" username and specify the above location of your private key file as means of authentication. 
You should then be able to connect with this profile to msuhp9 WITHOUT password authentication! 

\subsubsection{Tortoise}
Install the subversion-client Tortoise at http://tortoisesvn.net/downloads 
Create a directory to hold the CPFEM subroutine on your PC. 

Right-Click in this folder and select "SVN checkout" from the context menu. Specify 
svn+ssh://msuhp9/home/svn/repos/cpfem

as the URL of the desired repository. This will use the profile named "msuhp9" from PuTTY and should hence not ask for any authentication from your end.

\subsection{Linux workstations}

\subsubsection{Key authentication}
if not already done, generate a 2048 bit RSA key pair using 
ssh-keygen -t rsa -b 2048 
and go for the standard options offered. This will create "id\_rsa" (private key) and "id\_rsa.pub" (public key) within your \verb|~/.ssh| folder.

Append \verb|id_rsa.pub| to \verb|~/.ssh/authorized_keys| and try logging into another workstation with 
ssh MPIE\\\\myName@msuwsX
(exchange X with 2...11). It should NOT require password authentication.

\subsubsection{Checkout}
create a directory to hold the subversion-controlled CPFEM routine and change into this. 
svn checkout svn+ssh://MPIE\\\\myName@msuhp9.mpie.de/home/svn/repos/cpfem
to copy the repository content to the current working directory -- done.

familiarize yourself with svn: svn help