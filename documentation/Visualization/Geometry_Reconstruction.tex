\documentclass[a4paper,12pt]{article}
\usepackage{bm}
\usepackage{mathbbol} %poor substitute for mathbb. Changed to mathbbg to have an option for greek bb
\usepackage{amssymb}
\usepackage{amsmath}
\usepackage{amsfonts}
\usepackage{graphicx}

\newcommand{\term}[1]{\textsc{#1}}
\newcommand{\eref}[1]{eq.~\eqref{#1}}
\newcommand{\Eref}[1]{Eq.~\eqref{#1}}
\newcommand{\fref}[1]{fig.~\ref{#1}}
\newcommand{\Fref}[1]{Fig.~\ref{#1}}
\newcommand{\tref}[1]{tab.~\ref{#1}}
\newcommand{\Tref}[1]{Tab.~\ref{#1}}
\newcommand{\cref}[1]{chapter~\ref{#1}}
\newcommand{\Cref}[1]{Chapter~\ref{#1}}
\newcommand{\sref}[1]{section~\ref{#1}}
\newcommand{\Sref}[1]{Section~\ref{#1}}
\newcommand{\lref}[1]{listing~\ref{#1}}
\newcommand{\Lref}[1]{Listing~\ref{#1}}

\newcommand{\ie}{i.e.}
\newcommand{\eg}{e.g.}
\newcommand{\cf}{cf.}

\newcommand{\field}[1]{\ensuremath{\mathcal{#1}}}
\newcommand{\tnsr}[1]{\ensuremath{\bm{{#1}}}}
\newcommand{\tnsrfour}[1]{\ensuremath{\mathbb{#1}}}
\newcommand{\gammaop}{\ensuremath{\mathbbg{\Gamma}}}
\newcommand{\abs}[1]{\ensuremath{\left|{#1}\right|}}
\newcommand{\norm}[1]{\ensuremath{\left|\left|{#1}\right|\right|}}
\newcommand{\vctr}[1]{\ensuremath{\bm{#1}}}
\newcommand{\inc}[1]{\ensuremath{{\rm d}{#1}}}
\newcommand{\sign}[1]{\ensuremath{\operatorname{sign}\left({#1}\right)}}
\newcommand{\grad}[1][]{\ensuremath{\operatorname{grad}{#1}}}
\newcommand{\divergence}[1][]{\ensuremath{\operatorname{div}{#1}}}
\newcommand{\totalder}[2]{\ensuremath{\frac{\inc{#1}}{\inc{#2}}}}
\newcommand{\partialder}[2]{\ensuremath{\frac{\partial{#1}}{\partial{#2}}}}

\newcommand{\identity}{\ensuremath{\tnsr I}}
\newcommand{\F}[1][]{\ensuremath{\tnsr F^{{\rm #1}}}}
\newcommand{\Favg}{{\ensuremath{\overline{\F}^{(n)}}}}
\newcommand{\Fp}[1][]{\ensuremath{\tnsr F_\text{p}^{#1}}}
\newcommand{\Fe}[1][]{\ensuremath{\tnsr F_\text{e}^{#1}}}
\newcommand{\Lp}{\ensuremath{\tnsr L_\text{p}}}
%\newcommand{\Q}[1]{\ensuremath{\tnsr Q^{(#1)}}}
%\newcommand{\x}[2][]{\ensuremath{\vctr x^{(#2)}_\text{#1}}}
%\newcommand{\y}[2][]{\ensuremath{\vctr x^{(#2)}_\text{#1}}}
\newcommand{\dg}[2][]{\ensuremath{\Delta\vctr g^{(#2)}_\text{#1}}}
\newcommand{\g}[1][]{\ensuremath{\vctr g_\text{#1}}}
%\newcommand{\A}[2][]{\ensuremath{A^{(#2)}_\text{#1}}}
\newcommand{\N}[2]{\ensuremath{\varrho^{(#1)}_\text{#2}}}
\newcommand{\Burgers}[1]{\ensuremath{\vctr s^{(#1)}}}
\newcommand{\n}[1]{\ensuremath{\vctr n^{(#1)}}}
\newcommand{\m}[2]{\ensuremath{\vctr m^{(#1)}_{#2}}}
\newcommand{\ld}[1]{\ensuremath{\vctr p^{(#1)}}}
\newcommand{\velocity}[2]{\ensuremath{v^{(#1)}_\text{#2}}}
\newcommand{\avgvelocity}[2]{\ensuremath{{\overline v}^{(#1)}_ \text{#2}}}
\newcommand{\flux}[2]{\ensuremath{\vctr f^{(#1)}_ \text{#2}}}
\newcommand{\averageflux}[2]{\ensuremath{\overline{\vctr f}^{(#1)}_ \text{#2}}}
\newcommand{\interfaceflux}[2]{\ensuremath{\tilde{\vctr f}^{(#1)}_ \text{#2}}}
\newcommand{\transmissivity}[1]{\ensuremath{\chi^{(#1)}}}
\newcommand{\galpha}{\ensuremath{\gamma^{(\alpha)}}}
\newcommand{\dotgalpha}{\ensuremath{\dot{\gamma}^{(\alpha)}}}
\newcommand{\taualpha}{\ensuremath{\tau^{(\alpha)}}}
\newcommand{\taualphamax}{\ensuremath{\hat\tau^{(\alpha)}}}
\newcommand{\density}[2]{\ensuremath{\varrho^{(#1)}_ \text{#2}}}
\newcommand{\densityfunc}[2]{\ensuremath{{\tilde\varrho}^{(#1)}_ \text{#2}}}
\newcommand{\avgdensity}[2]{\ensuremath{{\overline\varrho}^{(#1)}_ \text{#2}}}
\newcommand{\dotdensity}[2]{\ensuremath{\dot{\varrho}^{(#1)}_ \text{#2}}}
\newcommand{\densityexcess}[2]{\ensuremath{\Delta\varrho^{(#1)}_ \text{#2}}}
\newcommand{\cs}[2][]{\ensuremath{\sigma^{(#1)}_ \text{#2}}}
\title{Geometry reconstruction using Fast Fourier transform}
\date{\today}

\author{Martin Diehl}
\begin{document}
\maketitle

The presented method allows the shape reconstruction of a volume element with periodic boundary conditions.
The deformation gradient on each point of a regular, three-dimensional lattice in undeformed configuration must be known.

The deformation gradient maps each point (or a line in the infinitesimal neighborhood of the point) into the current configuration.
It is defined as:
\begin{equation}
\F(\vctr x) = \left(
   \begin{array}{ccc}
     \frac{\partial y_1}{\partial x_1} & \frac{\partial y_1}{\partial x_2} & \frac{\partial y_1}{\partial x_3} \\
     \frac{\partial y_2}{\partial x_1} & \frac{\partial y_2}{\partial x_2} & \frac{\partial y_2}{\partial x_3} \\
     \frac{\partial y_3}{\partial x_1} & \frac{\partial y_3}{\partial x_2} & \frac{\partial y_3}{\partial x_3} \\
   \end{array}
\right)
\end{equation}
with \vctr y\ are the coordinates in current configuration and \vctr x\ are the coordinates in reference configuration.

The three-dimensional field of second order tensors is transformed to the Fourier space, giving three-dimensional field of second order tensors that depend on the three dimensional wave vector and not on the vector of spatial coordinates:
\begin{equation}
\mathcal F \left( \F(\vctr x) \right)=  \hat{\F}(\vctr k)
\end{equation}

Integration in Fourier space works is defined for the one dimensional case as:
\begin{equation}
\mathcal{F} \left( \int_{-\infty}^{x} g (\tau) d \tau \right) = \frac{\mathcal{F}{g(x)}}{i2 \pi k} + c \delta(k)
\end{equation}
where the last term is the integration constant.
Constant (or linear after integration) terms cannot properly handled when using the integration property of the Fourier domain, as a division by the ``constant wave'' ($k=0$) is not possible.
Thus, to carry out the integration the function is separated into an average and a locally fluctuating part.
The locally fluctuating part is integrated in Fourier space, while the integration of the constant part is easier in the real domain.

The fluctuation field of the position vector in deformed configuration in Fourier space reads as:
\begin{equation}
\hat{\tilde{y}}_{\rm j}(\vctr k) = \hat{F_{\rm ji}}(\vctr k) \left(k_{\rm i} i 2 \pi \right)^{-1} \forall k_i \neq 0
\end{equation}
The average part is set to zero in Fourier space.

The inverse Fourier transform gives the locally fluctuating part of each position in current configuration:
\begin{equation}
\mathcal{F}^{-1}\left(\hat{\tilde{\vctr y}}(\vctr k) \right) = \tilde{\vctr y}(\vctr x)
\end{equation}
and the position vector in undeformed configuration is given as:
\begin{equation}
\vctr y(\vctr x) = \overline{\F}: \vctr x + \tilde{\vctr y}(\vctr x)
\end{equation}

\end{document}
