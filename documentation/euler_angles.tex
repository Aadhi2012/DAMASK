\chapter{Crystallographic orientations}
\section{Bunge Euler angles}
\label{bunges}
Euler angles (\beaphiI, \beaPhi , \beaphiII) -- in Bunge convention -- rotate the sample coordinate system ($X$,~$Y$,~$Z$  or RD,~TD,~ND) into the crystal coordinate system ($x_c$, $y_c$, $z_c$). 
Three successive rotations are carried out in the following way \citep[pg.\,4]{Bunge1982}:
\begin{enumerate}
	\item Rotate by angle $\pI$ around Z, to bring X into the $x_c$--$y_c$-plane. The new intermediate axes are X', Y' and Z (Z is unchanged).
	\item Now rotate $\beaPhi$ degrees around X', to make Z parallel with $z_c$. The intermediate axes are X', Y'', Z'.
	\item A rotation by angle $\pII$ around Z' makes the rotated axes then identical to the crystal axes.
\end{enumerate}

The rotation matrix can be calculated as
\[% Gottstein pg 55
\bsym{g}=\left(\begin{array}{ccc}
\cos{\pI}\cos{\pII}-\sin{\pI}\sin{\pII}\cos{\beaPhi} & \sin{\pI}\cos{\pII}+\cos{\pI}\sin{\pII}\cos{\beaPhi}& \sin{\pII}\sin{\beaPhi}\\
-\cos{\pI}\sin{\pII}-\sin{\pI}\cos{\pII}\cos{\beaPhi} & -\sin{\pI}\cos{\pII}+\cos{\pI}\cos{\pII}\cos{\beaPhi}& \cos{\pII}\sin{\beaPhi}\\
\sin{\pI}\sin{\beaPhi} & -\cos{\pI}\sin{\beaPhi}& \cos{\beaPhi}
\end{array}\right)
\]

